% Appendix C

\chapter{Python code for parsing CPU usage} % Main appendix title
\label{AppendixC}
\section{How to parse the output from Busybox}
\begin{minipage}{\linewidth}
\begin{lstlisting}[caption={A sample of five seconds fragment output from the text file}, label = {listing:BUSYBOXTOPTEXT}, captionpos=b, basicstyle=\ttfamily\footnotesize]
[H[JMem: 2720896K used, 39068K free, 0K shrd, 2496K buff, 537K cached
CPU: 27.2% usr  3.6% sys  0.0% nic 67.7% idle  0.2% io  0.6% irq  0.3% sirq
Load average: 6.89 6.25 5.88 3/2638 12713
[7m  PID  PPID USER     STAT   VSZ %VSZ CPU %CPU COMMAND[0m
11161   769 app_199  S <  2310m 85.3   0 25.6 {i.viettt.mscosa} no.uio.ifi.viett
  585     1 system   R <   133m  4.9   1  1.8 /system/bin/surfaceflinger
 1521   769 system   S    2530m 93.4   0  1.1 system_server
10096     2 root     SW       0  0.0   0  0.3 [kworker/u8:3]
  602     1 system   S    68152  2.4   2  0.3 /system/vendor/bin/mm-pp-dpps
12592     2 root     DW       0  0.0   0  0.3 [mdss_fb0]
  578     1 system   S     9424  0.3   0  0.3 /system/bin/servicemanager
10861     2 root     DW       0  0.0   0  0.3 [kworker/u8:5]
    7     2 root     SW       0  0.0   3  0.2 [rcu_preempt]
 2021   769 system   S <  2417m 89.2   2  0.1 {ndroid.systemui} com.android.syst
12706 23916 shell    R     1416  0.0   3  0.1 busybox top d 5
....................
\end{lstlisting}
\end{minipage}
To get the percent CPU usage from the text files, a pattern for identifying the percent numbers must be identified. As presented in Listing \ref{listing:BUSYBOXTOPTEXT}, the numbers are always in between a space and an other space following by \{i.viettt.mscosa\}. Hence, all percent CPU usage for each 5 second fragment can be gotten by scanning the text files based on the pattern.
\section{Python code for parsing}
Python is a high-level programming language that is a good candidate for parsing text and drawing charts. To parse and draw charts from the text files, three libraries are used, which are re (regular expression), numpy (the fundamental package for scientific computing with Python), and matplotlib.pyplot (drawing chart). The text files are read into buffers, then the pattern is applied on each buffer. The results from applying the pattern are presented on a chart by using pyplot library. The python code for parsing and drawing the percent CPU usages is presented in \ref{listing:PYTHONCPU}.
\begin{minipage}{\linewidth}
\begin{lstlisting}[caption={Python code for parsing CPU usage}, label = {listing:PYTHONCPU}, language = python ,captionpos=b, basicstyle=\ttfamily\footnotesize]
import sys
import re
import numpy
import matplotlib.pyplot as pp

if __name__ == "__main__":
    #Get param from terminal
    filetext = sys.argv[1]
    filetext1 = sys.argv[2]
    filetext2 = sys.argv[3]

    f = open(filetext)
    content_source = f.read()
    f1 = open(filetext1)
    content_source1 = f1.read()
    f2 = open(filetext2)
    content_source2 = f2.read()
    f.close()
    f1.close()
    f2.close()

    matches = [float(i) for i in re.findall('\S[\d+\.]*\d+(?=\s+\{i\.viettt)', content_source, re.DOTALL)]
    matches1 = [float(i) for i in re.findall('\S[\d+\.]*\d+(?=\s+\{i\.viettt)', content_source1, re.DOTALL)]
    matches2 = [float(i) for i in re.findall('\S[\d+\.]*\d+(?=\s+\{i\.viettt)', content_source2, re.DOTALL)]
    
    print(numpy.mean(matches))
    print(numpy.mean(matches1))
    print(numpy.mean(matches2))

    pp.plot(numpy.arange(0, len(matches)*5,5), matches)
    pp.plot(numpy.arange(0, len(matches1)*5,5), matches1)
    pp.plot(numpy.arange(0, len(matches2)*5,5), matches2)

    pp.legend(['buffer 10s', 'buffer 20s', 'buffer 30s'], loc='upper left')
    pp.xlabel('seconds')
    pp.ylabel('%CPU')
    pp.title('CPU USAGE')
    pp.show()
\end{lstlisting}
\end{minipage}